\section{Margins and spacing}
Example fonts in Latex could be found here\cite{latex-font}.\\

\begin{itemize}
    \item Left margin: 1.5” from left edge of page
    \item Right margin: 1.0” from right edge of page
    \item Top margin: 1.0” from top edge of page
    \item Bottom margin: 1.25” from bottom edge of page 
\end{itemize}

\begin{table}[]
\centering
\caption{Example of a table.}
\label{tab:complexitylocal}
\begin{tabular}{|l|l|}
\hline
\textbf{Year} & \textbf{Name} \\ \hline
\end{tabular}
\end{table}

\section{Fonts}
Fonts should be easy to read. Times New Roman, Arial or a similarly clear font is preferred; type size must be 10, 11, or 12 point. Script and italic typefaces are not acceptable except where absolutely necessary i.e. in Latin designations of species, etc. \\
In preparing your dissertation or thesis for electronic submission, you must embed all fonts. In Microsoft Word 2013, this is done by accessing the FILE menu; select OPTIONS, select SAVE. From the SAVE menu check the box by” Embed fonts in the file”. If the file size is a concern, check the box next to “Do NOT embed common system fonts”. \\
Large tables, charts, etc., may be reduced to conform to page size, but the print must remain clear enough to be readable. You can also attach a PDF for electronic submissions.

\section{Page numbering}
Every page, with the exception of the title page, the copyright page, and the committee approval page is numbered in the upper right hand corner, one half inch from the top of the page and one inch from the right edge of the page. Do not underline or place a period after the number. Do not use a running header. \\
\begin{itemize}
    \item The prefatory materials (abstract, acknowledgements, table of contents, etc.) are numbered in lower case Roman numerals (i, ii, iii, iv…). Insert a section break after the Roman numerals to create different page numbering styles.
    \item The first page of the main text and all subsequent pages are continuously numbered in Arabic numerals beginning with 1 until the final page number (1, 2, 3, 4…).
    \item Do NOT number appendices or pages of additional material with numbers such as 4a or A-1.
\end{itemize}

\section{Tables and appendices}
Tables and appendices are part of the document and must conform to the same margin and page numbering requirements.

\section{Sequence of pages}
Assemble pages in the following order:

\begin{itemize}
    \item Title page *no page number* (create according to example provided)
    \item Copyright Notice *no page number* (optional - see example)
    \item Committee Approval Page *no page number* (use\cite{unr-2020-forms} NO SIGNATURES on this page)
    \item Abstract (begins lower case Roman numerals i, ii, iii…)
    \item Dedication (optional)
    \item Acknowledgments (optional)
    \item Table of Contents
    \item List of Tables
    \item List of Figures
    \item Body of Manuscript (begins Arabic numbering 1, 2, 3…)
    \item Back Matter (appendices, notes, bibliography, etc.)
\end{itemize}

\section{Title page}
\begin{itemize}
    \item Do not number the title page
    \item Center each line of type
    \item Use BOLD text type for the manuscript title
    \item The date listed is the month and year in which you will graduate. The only acceptable months are May, August, and December (graduation cycles).
\end{itemize}

\section{Copyright page}
No page number on this page. Although not required, we strongly recommend you insert a copyright notice in your manuscript following the title page. Essential components of the copyright notice are: copyright symbol, full legal name of author, and year of first publication. Follow the format of the sample provided below.

\section{Committee approval page}
\begin{itemize}
    \item No page number on this page
    \item Use the electronic PDF template provided below. This page will list the advisory committee members and graduate dean but will NOT include committee signatures.
\end{itemize}

\section{Abstract}
Lower case Roman numeral ``i'' page number. \\

Abstracts are required for all theses and dissertations. ProQuest no longer has a word limit on the abstract, ``as this constrains your ability to describe your research in a section that is accessible to search engines, and therefore would constrain potential exposure of your work.'' ProQuest does publish print indices that include citations and abstracts of all dissertations and theses published by ProQuest/UMI. These print indices require word limits of 350 words for doctoral dissertations and 150 words for master’s theses (only text will be included in the abstract). You may wish to limit the length of your abstract if this concerns you. The abstracts as you submit it will NOT be altered in your published manuscript.

\section{Instructions for completing dissertation committee approval page}
Please follow the forms shared in: \href{https://www.unr.edu/grad/student-resources/filing-guidelines}{unr-grad/filing-guidelines}