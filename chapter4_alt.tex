These guidelines apply to those theses or dissertations which consist of a number of papers either previously published or being published concurrently with the submission of the thesis or dissertation. Acceptance and publication of the articles are not criteria for this alternative. Each of the papers should constitute a separate chapter of the overall work. Preceding the papers should be an introductory section. This section may be one or more chapters but should include:
\begin{itemize}
    \item an overall introduction to the thesis/dissertation,
    \item a review of the appropriate literature,
    \item and a description of methodology used in the study.
\end{itemize}

The student’s advisory committee should determine the format and specific content of this introductory section. \\

The number of individual papers constituting chapters of the thesis/dissertation is determined by the student’s advisory committee. These chapters may be formatted in the same style required by the journals to which they are to be submitted. However, the margins must conform to those of the overall thesis, i.e. left margin = 1.5"; right margin = 1"; top margin = 1"; bottom margin = 1.25". In addition, each page must be numbered consistent with the rest of the thesis/dissertation, that is, the first page of text is numbered 1 with each subsequent page numbered consecutively until the end, to include all appendices, indexes, etc. \\

Following the chapters consisting of individual papers, there must follow a summary, conclusions and recommendations section. This section may be formatted as one or more chapters. \\

Work reported in the articles should represent a major contribution by the student that is the review of the literature, the conceptual framework and/or research design for the reported work. The statistical analyses, summary, conclusions, and recommendations should represent the student’s own work. \\

For publication purposes, other researchers may be named as additional authors. This would be especially appropriate when publication is dependent upon extensive revision of the initial manuscript submitted and the faculty involved assumes responsibility for the revisions, or when the student is using an existing data base. \\


When a student chooses this option, the articles will be submitted to the journals agreed upon by the concerned academic unit. Responsibility for follow-up, revisions, etc., should be identified in a written document and agreed upon by the student and faculty member(s) involved. \\
a
